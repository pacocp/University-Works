\input{preambuloSimple.tex}
 \usepackage{algpseudocode}
%----------------------------------------------------------------------------------------
%	TÍTULO Y DATOS DEL ALUMNO
%----------------------------------------------------------------------------------------

\title{	
\normalfont \normalsize 
\textsc{{\bf Algorítmica (2015-2016)} \\ Grado en Ingeniería Informática \\ Universidad de Granada} \\ [25pt] % Your university, school and/or department name(s)
\horrule{0.5pt} \\[0.4cm] % Thin top horizontal rule
\huge Práctica 4-Segunda parte: TSP \\ % The assignment title
\horrule{2pt} \\[0.5cm] % Thick bottom horizontal rule
}

\author{Francisco Carrillo Pérez,Borja Cañavate Bordons, \\Miguel Porcel Jiménez,Jose Manuel Rejón Santiago,Jose Arcos Aneas} % Nombre y apellidos

\date{\normalsize\today} % Incluye la fecha actual

%----------------------------------------------------------------------------------------
% DOCUMENTO
%----------------------------------------------------------------------------------------

\begin{document}

\maketitle % Muestra el Título

\newpage %inserta un salto de página

\tableofcontents % para generar el índice de contenidos

\listoffigures

\listoftables

\newpage
	
	\section{Introducción }

		\begin{itemize}
			\item El objetivo de esta práctica es resolver el problema del TSP utilizando para ello un enfoque Branch and Bound y, alternativamente, otro con Backtracking y comparar ambos.
		\end{itemize}

	
	
	
	%------------------------------------------------
		\section{Representación de la solución} 
		
		
Vector de tamaño N en el que el índice del vector nos indica el orden en el que las ciudades deben de ser visitadas y el contenido de cada posición hace referencia al índice de la ciudad
	
	\section{Diseño del algoritmo: Branch and Bound} 


		
		Usaremos una cola con prioridad que será donde se introduzcan los nodos.
		En la cola se seleccionará el nodo con mayor prioridad, y de dicho nodo se consultará el valor de su cota local, que en caso de ser mejor a la global  y ser nodo hoja, se actualizará la solucion, en caso sólo de ser mejor que la global, se añadiran los hijos del nodo a la cola. En caso de ser peor, se devuelve la solucion actual.
		
		

	\section{Pseudocódigo}
	


		\footnotesize{	
			\begin{algorithmic}				
				\Require Matriz\_costes, Vector\_ciudades[N] Vector\_distancias\_minimas; 
				\State \textbf{Branch\&Bound( Matriz\_costes, Vector\_ciudades Vector\_distancias\_minimas):}
				\State priority\_queue cola;
				\State solucion\_final;
				\State	 Nodo n.generarnodo(Vector\_ciudades[0])
				\While {\textbf{!cola.empty()}}
				\State nodo = cola.top();
				\If{EsHoja(nodo)  \&\& (nodo.cotalocal > cota global ) }
				\State solucion\_final = nodo.solucion;
				\State cota global = nodo.cotalocal;
				\EndIf
				\If{(nodo.cotalocal < cota global )}
				\State cola.add(nodo.generarhijos())
				\State cola.push;
				\Else
				\State return solucion\_final;
				
				\EndIf
				
				\EndWhile
				
				
				
			\end{algorithmic}	
		}

	
	\section{Diseño del algoritmo: Backtracking} 


		
		La solución planteada es el mismo concepto que la aplicada en la anterior práctica, recorreremos el árbol en profundidad, calculando en cada nodo el valor de la cota local salvo que en caso de ser mayor que la global, podamos (ignoramos) a ese nodo y a todos sus hijos.
		
		

	\section{Pseudocódigo}


		\footnotesize{	
			\begin{algorithmic}				
				\Require Matriz, S\_final[N] S\_parcial[N] Ciudades[N]={false} ciudad\_actual, nivel,valor\_maximo=0;
				\State \textbf{Backtrack(S,S\_parcial,Ciudades,ciudad\_actual,nivel):}
				\State Ciudades[ciudad\_actual]=true;
				\State	   S\_parcial[nivel - 1]=ciudad\_actual;
				\For {\textbf{i} to \textbf{N}}
				
				\If {Ciudades[i]==false}
				\State 	valor\_actual = CalcularSolucionActual(S\_parcial);
				\If {(cotalocal < cotaglobal)}
				\State 	\textbf{Backtrack(S,S\_parcial,Ciudades,i,nivel+1);}
				\EndIf
				\If{nodo\_actual == nodo\_hoja}
				\State{ valor\_actual = CalcularSolucionActual(S\_parcial)}
				\If{valor\_actual \textbf{mayor que} valor\_maximo}
				\State{ S\_final = S\_Actual}
				\State{ valor\_maximo = valor\_actual}
				\EndIf
				\EndIf
				
				\State Ciudades[i] = false;
				\EndIf
				\EndFor
			\end{algorithmic}			
		}

	
	
	\section{Comparación}


		En ambos programas, hemos analizado los mismos mapas. Compararemos los tiempos y el empleo de los nodos para cada tipo de técnica empleada

	
	


		\begin{figure}[H]
			\centering
			\includegraphics[width=0.7\linewidth]{../Codigo/Solucion/comparacion}
			\caption{comparacion entre ambos }
			\label{fig:comparacion}
		\end{figure}
		
		

	
	



		\begin{table}[H]
							Mapa ulysses 5
			\centering
			
			\label{my-label}
			\begin{tabular}{lllll}
				Técnica	& Branch and bound & Backtracking  &   \\
				N.totales& 120&120  &  &  \\
				N.podados&  35& 68 &  &  \\
				N.explorados&85  &52  &  &  \\
				
			\end{tabular}
		\end{table}
		

	
	



		\begin{table}[H]
			Mapa ulysses 8
			\centering
			
			\label{my-label}
			\begin{tabular}{lllll}
				Técnica	& Branch and bound & Backtracking  &   \\
				N.totales& 40320&40320  &  &  \\
				N.podados&  38568& 37605 &  &  \\
				N.explorados&1752  &2715 &  &  \\
				
			\end{tabular}
		\end{table}
		

	
	


 
		\begin{table}[H]
							Mapa ulysses 10
			\centering
			
			\label{my-label}
			\begin{tabular}{lllll}
				Técnica	& Branch and bound & Backtracking  &   \\
				N.totales& 3628800&3628800  &  &  \\
				N.podados&  3601453& 3566077 &  &  \\
				N.explorados&27347  &62723  &  &  \\
				
			\end{tabular}
		\end{table}
		

	
	

	
						
		\begin{table}[H]
			Mapa ulysses 12
			\centering
			
			\label{my-label}
			\begin{tabular}{lllll}
				Técnica	& Branch and bound & Backtracking  &   \\
				N.totales& 479001600&479001600 &  &  \\
				N.podados&  478284703& 474474230&  &  \\
				N.explorados&716897  &4527370  &  &  \\
				
			\end{tabular}
		\end{table}
		

	


\end{document}