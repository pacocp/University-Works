\input{preambuloSimple.tex}
 \usepackage{algpseudocode}
%----------------------------------------------------------------------------------------
%	TÍTULO Y DATOS DEL ALUMNO
%----------------------------------------------------------------------------------------

\title{	
\normalfont \normalsize 
\textsc{{\bf Algorítmica (2015-2016)} \\ Grado en Ingeniería Informática \\ Universidad de Granada} \\ [25pt] % Your university, school and/or department name(s)
\horrule{0.5pt} \\[0.4cm] % Thin top horizontal rule
\huge Práctica 4-Primera Parte: Cena de gala \\ % The assignment title
\horrule{2pt} \\[0.5cm] % Thick bottom horizontal rule
}

\author{Francisco Carrillo Pérez,Borja Cañavate Bordons, \\Miguel Porcel Jiménez,Jose Manuel Rejón Santiago,Jose Arcos Aneas} % Nombre y apellidos

\date{\normalsize\today} % Incluye la fecha actual

%----------------------------------------------------------------------------------------
% DOCUMENTO
%----------------------------------------------------------------------------------------

\begin{document}

\maketitle % Muestra el Título

\newpage %inserta un salto de página

\tableofcontents % para generar el índice de contenidos

\listoffigures

\listoftables

\newpage

\section{Introducción }

El objetivo de esta práctica es diseñar un algoritmo Backtracking, que resuelva uno de los cinco problemas de la práctica y realizar un estudio empírico de su eficiencia.
	
	Se desea sentar a N invitados alrededor de una mesa, de manera que cada invitado tendra a su lado a otros dos. Cada par de invitados tiene un nivel de compatibilidad. Se desea maximizar la compatibilidad de estos comensales.
%----------------------------------------------------------------------------------------

	

%------------------------------------------------------------------------------------------

\section{Elementos de la solución al problema}

	Dada una matriz M[i][j] 
	Mantenemos en la matriz la afinidad  entre 
	el comensal i y el comensal j
	
	
	\[ \left( \begin{array}{ccc}
	0 & 30 & 15 \\
	30 & 0 & 20 \\
	15 & 20 & 0 \end{array} \right)\] 



\subsection{Representación de la compatibilidad}
La entrada sera una matriz simetrica de valores aleatorios con la diagonal de 0s.

\subsection{Representación de la solución}
Vector de longitud igual al número de invitados (\textit{N}), en que cada posición guarda el valor del invitado que se sienta en la posición \textit{i}.

\subsection{Solucion parcial}
olucion parcial al problema de tamaño menor que N.


\subsection{Restricciones explícitas}
Los valores que puede tomar la solucion son los enteros de 1 a N. Donde N es el número total de invitados. 

\subsection{Restricciones implícitas}
Estas restricciones son las que determinan si una función parcial puede llevarnos a una solucion del problema. Si supera un umbral. 


%--------------------------------------------------------------------------------------------

\section{Pseudocódigo}

\begin{algorithmic}				
	\Require Matriz, S\_final[N] S\_parcial[N] Sentados[N]={false} comensal\_actual, nivel,valor\_maximo=0;
	\State \textbf{Funcion(S,S\_parcial,Sentados,comensal\_actual,nivel):}
	\State Sentados[comensal\_actual]=true;
	\State	   S\_parcial[nivel - 1]=comensal\_actual;
	\For {\textbf{i} to \textbf{N}}
	
	\If {Sentados[i]==false}
	\State 	valor\_actual = CalcularSolucionActual(S\_parcial);
	\State 	\textbf{Funcion(S,S\_parcial,Sentados,i,nivel+1);}
	\If{nodo\_actual == nodo\_hoja}
	\State{ valor\_actual = CalcularSolucionActual()}
	\If{valor\_actual \textbf{mayor que} valor\_maximo)}
	\State{ S\_final = S\_Actual}
	\State{ valor\_maximo = valor\_actual}
	\EndIf
	\EndIf
	
	\State Sentados[i] = false;
	\EndIf
	\EndFor
	
	
\end{algorithmic}	


\section{Eficiencia}




	
	\begin{figure}[H]
		\centering
		\includegraphics[width=0.7\linewidth,scale=1.5]{../Codigo/backtrackempirica}
		\caption{algoritmo backtracking}
		\label{fig:backtrackhibrida}
	\end{figure}
	
	

\subsection{Eficiencia Híbrida}	
	
	\begin{figure}[H]
		\centering
		\includegraphics[width=0.7\linewidth,scale=1.5]{../Codigo/backtrackhibrida}
		\caption{comparación x**8 vs algoritmo backtracking}
		\label{fig:backtrackhibrida}
	\end{figure}
	

	Ajuste con X**8
	
	\begin{table}[H]
		\centering
		\caption{Ef híbrida obtenida}
		\label{my-label}
		\begin{tabular}{lll}
			& Final set of parameters &   Asymptotic Standard Error & \\
			& a0  = 8.59555e-09 &  +/- 2.326e-11    (0.2706\%)&  \\
			
		\end{tabular}
	\end{table}
	




\end{document}