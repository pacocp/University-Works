%%%%%%%%%%%%%%%%%%%%%%%%%%%%%%%%%%%%%%%%%
% Short Sectioned Assignment LaTeX Template Version 1.0 (5/5/12)
% This template has been downloaded from: http://www.LaTeXTemplates.com
% Original author:  Frits Wenneker (http://www.howtotex.com)
% License: CC BY-NC-SA 3.0 (http://creativecommons.org/licenses/by-nc-sa/3.0/)
%%%%%%%%%%%%%%%%%%%%%%%%%%%%%%%%%%%%%%%%%

%----------------------------------------------------------------------------------------
%	PACKAGES AND OTHER DOCUMENT CONFIGURATIONS
%----------------------------------------------------------------------------------------

\documentclass[paper=a4, fontsize=11pt]{scrartcl} % A4 paper and 11pt font size

% ---- Entrada y salida de texto -----

\usepackage{hyperref}
\usepackage{varioref}
\usepackage[T1]{fontenc} % Use 8-bit encoding that has 256 glyphs
\usepackage[utf8]{inputenc}
%\usepackage{fourier} % Use the Adobe Utopia font for the document - comment this line to return to the LaTeX default

% ---- Idioma --------

\usepackage[spanish, es-tabla]{babel} % Selecciona el español para palabras introducidas automáticamente, p.ej. "septiembre" en la fecha y especifica que se use la palabra Tabla en vez de Cuadro

% ---- Otros paquetes ----

\usepackage{amsmath,amsfonts,amsthm} % Math packages
%\usepackage{graphics,graphicx, floatrow} %para incluir imágenes y notas en las imágenes
\usepackage{graphics,graphicx, float} %para incluir imágenes y colocarlas

% Para hacer tablas comlejas
%\usepackage{multirow}
%\usepackage{threeparttable}

%\usepackage{sectsty} % Allows customizing section commands
%\allsectionsfont{\centering \normalfont\scshape} % Make all sections centered, the default font and small caps

\usepackage{fancyhdr} % Custom headers and footers
\pagestyle{fancyplain} % Makes all pages in the document conform to the custom headers and footers
\fancyhead{} % No page header - if you want one, create it in the same way as the footers below
\fancyfoot[L]{} % Empty left footer
\fancyfoot[C]{} % Empty center footer
\fancyfoot[R]{\thepage} % Page numbering for right footer
\renewcommand{\headrulewidth}{0pt} % Remove header underlines
\renewcommand{\footrulewidth}{0pt} % Remove footer underlines
\setlength{\headheight}{13.6pt} % Customize the height of the header

\numberwithin{equation}{section} % Number equations within sections (i.e. 1.1, 1.2, 2.1, 2.2 instead of 1, 2, 3, 4)
\numberwithin{figure}{section} % Number figures within sections (i.e. 1.1, 1.2, 2.1, 2.2 instead of 1, 2, 3, 4)
\numberwithin{table}{section} % Number tables within sections (i.e. 1.1, 1.2, 2.1, 2.2 instead of 1, 2, 3, 4)

\setlength\parindent{0pt} % Removes all indentation from paragraphs - comment this line for an assignment with lots of text

\newcommand{\horrule}[1]{\rule{\linewidth}{#1}} % Create horizontal rule command with 1 argument of height


%----------------------------------------------------------------------------------------
%	TÍTULO Y DATOS DEL ALUMNO
%----------------------------------------------------------------------------------------

\title{	
	\normalfont \normalsize 
	\textsc{{\bf Ingeniería de Servidores (2015-2016)} \\ Grado en Ingeniería Informática \\ Universidad de Granada} \\ [25pt] % Your university, school and/or department name(s)
	\horrule{0.5pt} \\[0.4cm] % Thin top horizontal rule
	\huge ISE CheatSheet \\ % The assignment title
	\horrule{2pt} \\[0.5cm] % Thick bottom horizontal rule
}

\author{Francisco Carrillo Pérez} % Nombre y apellidos

\date{\normalsize\today} % Incluye la fecha actual

%----------------------------------------------------------------------------------------
% DOCUMENTO
%----------------------------------------------------------------------------------------

\begin{document}
	
	\maketitle % Muestra el Título
	
	\newpage %inserta un salto de página
	
	\tableofcontents % para generar el índice de contenidos
	
	\newpage
	
\section{Variables}

\subsection{Variables temporales de una estación de servicio}
\begin{itemize}
	\item \textbf{Tiempo de espera en cola(\textit{W,waiting time})}
	\item \textbf{Tiempo de servicio(\textit{S,service time})}
	\item \textbf{Tiempo de respuesta de la estación de servicio(\textit{R,response time})}:\\
	$R=W+S$
\end{itemize}
\subsection{Redes de colas cerradas}
\begin{itemize}
	\item \textbf{Red cerrada tipo batch: } $N_T=N_0$
	\item \textbf{Red cerrada tipo interactivo: } $N_T=N_0+N_Z$
	\item \textit{$N_0=$Número de trabajos en el servidor}
	\item \textit{$N_Z=$ Número de trabajos en reflexión (esperando a que los usuarios vuelvan a introducirlos en el servidor)}
\end{itemize}
\subsection{Variables operacionales básicas}
\begin{itemize}
	\item \textbf{$T$} Duración del periodo de medida para el que se extrae el modelo.
	\item \textbf{$A_i$} Número de trabajos solicitados a la estación (llegadas, \textit{arrivals})
	\item \textbf{$C_i$} Número de trabajos completados por la estación (salidas,\textit{completions})
	\item \textbf{$B_i$} Tiempo que el dispositivo está ocupado (\textit{busy time})
\end{itemize}
\subsection{Variables operacionales deducidas}
\begin{itemize}
	\item \textbf{$\lambda_i$} Tasa de llegada (\textit{arrival rate}): $\lambda_i = \frac{A_i}{T} $ trabajos/segundo
	\item \textbf{$\tau_i$} Tiempo medio entre llegadas (\textit{interarrival time}): $\tau_i= \frac{1}{\lambda_i}=\frac{T}{A_i}$ segundos[/trabajo]
	\item \textbf{$X_i$} Productividad(\textit{throughput}): $X_i=\frac{C_i}{T}$ trabajos/segundo
	\item \textbf{$S_i$} Tiempo medio de servicio (\textit{service time}): $S_i=\frac{B_i}{C_i}$ segundos
	\item \textbf{$W_i$} Tiempo medio de espera en cola (\textit{waiting time}): $W_i=R_i - S_i$ segundos
	\item \textbf{$R_i$} Tiempo medio de respuesta (\textit{response time}): $R_i= W_i + S_i$ segundos
	\item \textbf{$N_i$} Número medio de trabajos en la estación de servicio
	\item \textbf{$Q_i$} Número medio de trabajos en cola de espera(\textit{jobs in queue})
	\item \textbf{$U_i$} Número medio de trabajos siendo servidos por el dispositivo: $U_i = N_i - Q_i$
\end{itemize}
\subsection{Variables globales del servidor}
\begin{itemize}
	\item \textbf{$A_0$} Número de trabajos solicitados al servidor (\textit{arrivals})
	\item \textbf{$C_0$} Número de trabajos completados por el servidor (\textit{completions})
	\item \textbf{$\lambda_0$} Tasa de llegada al servidor (\textit{arrival rate}): $\lambda_i = \frac{A_0}{T} $ trabajos/segundo
	\item \textbf{$\tau_0$} Tiempo medio entre llegadas al servidor (\textit{interarrival time}): $\tau_0= \frac{1}{\lambda_0}=\frac{T}{A_0}$ segundos[/trabajo]
	\item \textbf{$X_0$} Productividad sel servidor (\textit{throughput}): $X_0=\frac{C_0}{T}$ trabajos/segundo
	\item \textbf{$R_0$} Tiempo medio de respuesta del servidor (\textit{response time}): $R_0= W_i + S_i$ segundos
	\item \textbf{$N_0$} Número medio de trabajos en el servidor (\textit{jobs}) = $N_1+N_2+...+N_k$
\end{itemize}
\subsection{Razón de visita y demanda de servicio}
\begin{itemize}
	\item \textbf{Razón de visita $V_i$} \textit{visit ratio}: $V_i= \frac{C_i}{C_0}$
	\item \textbf{Demanda de servicio $D_i$} \textit{service demand}: $D_i= \frac{B_i}{C_0}= V_i \times S_i$
\end{itemize}

\section{Leyes}
\subsection{Ley de Little}
\textbf{Aplicada a un servidor: } $N_0=\lambda_0 \times R_0$\\
\textbf{Bajo la hipótesis del equilibrio de flujo(servidor no saturado): }\\ $N_0=\lambda_0 \times R_0 = X_0 \times R_0$\\
\textbf{Aplicación a toda una estación de servicio: } $N_i=\lambda_i \times R_i = X_i \times R_i$\\
\textbf{Aplicación a la cola de una estación de servivio: } $Q_i=\lambda_i \times W_i = X_i \times W_i$\\
\subsection{Ley de la Utilización}
\textbf{Aplicada a un servidor: } $U_i=\frac{B_i}{T}= \frac{C_i}{T} \times \frac{B_i}{C_i} = X_i \times S_i$\\
\textbf{Para un sistema en equilibrio de flujo(servidor no saturado): }\\ $U_i=\lambda_i \times S_i=X_i \times S_i$
\subsection{Ley de flujo forzado}
\textbf{Ley de Flujo Forzado: } $X_i=X_0 \times V_i$\\
\textbf{Relación Utilización-Demanda de Servicio: } $U_i=X_i \times S_i=X_0 \times V_i \times S_i = X_0 \times D_i$
\subsection{Ley General del tiempo de respuesta}
$R_0 = V_1 \times R_1 + V_2 \times R_2+...+ V_k \times R_k = \sum_{i=1}^k V_i \times R_i $
\subsection{Ley del tiempo de respuesta interactivo}
$R_0 =\frac{N_T}{X_0} - Z$
\end{document}